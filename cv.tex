%-----------------------------------------------------------------------------------------------------------------------------------------------%
%	The MIT License (MIT)
%
%	Copyright (c) 2021 Jitin Nair
%
%	Permission is hereby granted, free of charge, to any person obtaining a copy
%	of this software and associated documentation files (the "Software"), to deal
%	in the Software without restriction, including without limitation the rights
%	to use, copy, modify, merge, publish, distribute, sublicense, and/or sell
%	copies of the Software, and to permit persons to whom the Software is
%	furnished to do so, subject to the following conditions:
%	
%	THE SOFTWARE IS PROVIDED "AS IS", WITHOUT WARRANTY OF ANY KIND, EXPRESS OR
%	IMPLIED, INCLUDING BUT NOT LIMITED TO THE WARRANTIES OF MERCHANTABILITY,
%	FITNESS FOR A PARTICULAR PURPOSE AND NONINFRINGEMENT. IN NO EVENT SHALL THE
%	AUTHORS OR COPYRIGHT HOLDERS BE LIABLE FOR ANY CLAIM, DAMAGES OR OTHER
%	LIABILITY, WHETHER IN AN ACTION OF CONTRACT, TORT OR OTHERWISE, ARISING FROM,
%	OUT OF OR IN CONNECTION WITH THE SOFTWARE OR THE USE OR OTHER DEALINGS IN
%	THE SOFTWARE.
%	
%
%-----------------------------------------------------------------------------------------------------------------------------------------------%

%----------------------------------------------------------------------------------------
%	DOCUMENT DEFINITION
%----------------------------------------------------------------------------------------

% article class because we want to fully customize the page and not use a cv template
\documentclass[a4paper,12pt]{article}

%----------------------------------------------------------------------------------------
%	FONT
%----------------------------------------------------------------------------------------

% % fontspec allows you to use TTF/OTF fonts directly
% \usepackage{fontspec}
% \defaultfontfeatures{Ligatures=TeX}

% % modified for ShareLaTeX use
% \setmainfont[
% SmallCapsFont = Fontin-SmallCaps.otf,
% BoldFont = Fontin-Bold.otf,
% ItalicFont = Fontin-Italic.otf
% ]
% {Fontin.otf}

%----------------------------------------------------------------------------------------
%	PACKAGES
%----------------------------------------------------------------------------------------
\usepackage{url}
\usepackage{parskip} 	

%other packages for formatting
\RequirePackage{color}
\RequirePackage{graphicx}
\usepackage[usenames,dvipsnames]{xcolor}
\usepackage[scale=0.9]{geometry}

%tabularx environment
\usepackage{tabularx}

%for lists within experience section
\usepackage{enumitem}

% centered version of 'X' col. type
\newcolumntype{C}{>{\centering\arraybackslash}X} 

%to prevent spillover of tabular into next pages
\usepackage{supertabular}
\usepackage{tabularx}
\newlength{\fullcollw}
\setlength{\fullcollw}{0.47\textwidth}

%custom \section
\usepackage{titlesec}				
\usepackage{multicol}
\usepackage{multirow}

%CV Sections inspired by: 
%http://stefano.italians.nl/archives/26
\titleformat{\section}{\Large\scshape\raggedright}{}{0em}{}[\titlerule]
\titlespacing{\section}{0pt}{10pt}{10pt}

%for publications
\usepackage[style=authoryear,sorting=ynt, maxbibnames=2]{biblatex}

%Setup hyperref package, and colours for links
\usepackage[unicode, draft=false]{hyperref}
\definecolor{linkcolour}{rgb}{0,0.2,0.6}
\hypersetup{colorlinks,breaklinks,urlcolor=linkcolour,linkcolor=linkcolour}
\addbibresource{citations.bib}
\setlength\bibitemsep{1em}

%for social icons
\usepackage{fontawesome5}

%debug page outer frames
%\usepackage{showframe}


% job listing environments
\newenvironment{jobshort}[2]
    {
    \begin{tabularx}{\linewidth}{@{}l X r@{}}
    \textbf{#1} & \hfill &  #2 \\[3.75pt]
    \end{tabularx}
    }
    {
    }

\newenvironment{joblong}[2]
    {
    \begin{tabularx}{\linewidth}{@{}l X r@{}}
    \textbf{#1} & \hfill &  #2 \\[3.75pt]
    \end{tabularx}
    \begin{minipage}[t]{\linewidth}
    \begin{itemize}[nosep,after=\strut, leftmargin=1em, itemsep=3pt,label=--]
    }
    {
    \end{itemize}
    \end{minipage}    
    }



%----------------------------------------------------------------------------------------
%	BEGIN DOCUMENT
%----------------------------------------------------------------------------------------
\begin{document}

% non-numbered pages
\pagestyle{empty} 

%----------------------------------------------------------------------------------------
%	TITLE
%----------------------------------------------------------------------------------------

% \begin{tabularx}{\linewidth}{ @{}X X@{} }
% \huge{Your Name}\vspace{2pt} & \hfill \emoji{incoming-envelope} email@email.com \\
% \raisebox{-0.05\height}\faGithub\ username \ | \
% \raisebox{-0.00\height}\faLinkedin\ username \ | \ \raisebox{-0.05\height}\faGlobe \ mysite.com  & \hfill \emoji{calling} number
% \end{tabularx}

\begin{tabularx}{\linewidth}{@{} C @{}}
\Huge{A}\LARGE{DITYA }\Huge{N}\LARGE{AIR}\\[7.5pt]
\href{https://github.com/4aditya}{\raisebox{-0.05\height}\faGithub\ 4aditya} \ $|$ \ 
\href{https://linkedin.com/in/4aditya}{\raisebox{-0.05\height}\faLinkedin\ 4aditya} \ $|$ \ 
% \href{https://mysite.com}{\raisebox{-0.05\height}\faGlobe \ mysite.com} \ $|$ \ 
\href{mailto:nairaditya92@gmail.com}{\raisebox{-0.05\height}\faEnvelope \ nairaditya92@gmail.com} \ $|$ \ 
\href{tel:+918668401243}{\raisebox{-0.05\height}\faMobile \ +91 86684 01243} \\
\end{tabularx}

%----------------------------------------------------------------------------------------
% EXPERIENCE SECTIONS
%----------------------------------------------------------------------------------------

% %Interests/ Keywords/ Summary
% \section{Summary}
% This CV can also be automatically complied and published using GitHub Actions. For details, \href{https://github.com/jitinnair1/autoCV}{click here}.

%Experience
\section{Work Experience}

\begin{tabularx}{\linewidth}{@{} l r @{}}
\textbf{Web Developer Intern, J.R and Associates} & {July 4, 2022 – August 8, 2022} \\
\multicolumn{2}{@{}X@{}}{Borivali, Mumbai}\\[3.75pt]
\multicolumn{2}{@{}X@{}}{\begin{itemize}
    \item Developed interactive and responsive websites using HTML, CSS, JavaScript, PHP, and MySQL.
    \item Built dynamic web pages, handled user input, and managed backend data efficiently.
    \item Improved front-end design and database integration skills through real-world projects.
\end{itemize}}
\end{tabularx}


% \begin{joblong}{Designation}{Mar 2019 - Jan 2021}
% \item long long line of blah blah that will wrap when the table fills the column width
% \item again, long long line of blah blah that will wrap when the table fills the column width but this time even more long long line of blah blah. again, long long line of blah blah that will wrap when the table fills the column width but this time even more long long line of blah blah
% \end{joblong}
  
%Projects
\section{Projects}

\begin{tabularx}{\linewidth}{@{} l r @{}}
\href{https://vagabond.vercel.app}{\textbf{Vagabond Inc.}} & June 2024 \\[3.75pt]
\multicolumn{2}{@{}X@{}}{\begin{itemize}
    \item A full-stack travel discovery and booking platform for campers, trekkers, and adventure seekers across India. Built using the MERN stack with a seamless, responsive UI.
    \item Includes detailed destination pages, interactive booking forms with real-time validations, and a secure authentication system.
    \item Backend supports secure user data and booking storage, along with automatic downloadable receipts.
    \item Highlights handpicked offbeat locations curated for nature and travel enthusiasts.
\end{itemize}}
\end{tabularx}

\begin{tabularx}{\linewidth}{@{} l r @{}}
\href{https://driftlane.vercel.app}{\textbf{Driftlane}} & July 2025 \\[3.75pt]
\multicolumn{2}{@{}X@{}}{\begin{itemize}
    \item AI-powered adventure planner that generates personalized itineraries and recommendations based on user mood, group type, and budget.
    \item Frontend in Next.js (App Router) and Tailwind CSS with dark-glass UI, filters, and smooth animations.
    \item Backend in Flask serving ML models (TF-IDF + KNN, KMeans, RandomForest) trained on curated datasets.
    \item Itinerary generation via LLM (Gemini/ChatGPT), with dynamic image fetching from Brave and Wikipedia.
    \item Robust fallback handling, modular API design, and adaptive filtering across devices.
\end{itemize}}
\end{tabularx}

\begin{tabularx}{\linewidth}{@{} l r @{}}
\href{https://tryelle.vercel.app/}{\textbf{Tryelle}} & Aug 2025 \\[3.75pt]
\multicolumn{2}{@{}X@{}}{\begin{itemize}
    \item A browser-based virtual try-on application that uses MediaPipe to let users try different clothing outfits in real time using their webcam.
    \item Backend in Flask serving Python ML models for key features.
    \item No downloads, no installations — just open the browser and try on digital outfits instantly!
    \item Features include live webcam pose detection, realistic overlay of digital shirts, and a responsive design with Tailwind CSS.
    \item Built with HTML5, CSS3 (Tailwind), JavaScript (Vanilla), MediaPipe Pose for pose estimation, and Canvas API for rendering.
\end{itemize}}
\end{tabularx}
%----------------------------------------------------------------------------------------
%	EDUCATION
%----------------------------------------------------------------------------------------
\section{Education}
\begin{tabularx}{\linewidth}{@{}l X@{}}
2023 - 2026 & Bachelor of Engineering in Data Engineering, \textbf{Universal College of Engineering} \\
& \hfill GPA: 7.62, Kaman, Maharashtra \\
2020 - 2023 & Diploma in Computer Engineering, \textbf{Bhausaheb Vartak Polytechnic} \\
& \hfill Percentage: 79.87\%, Vasai, Maharashtra \\
\end{tabularx}
%----------------------------------------------------------------------------------------
%	PUBLICATIONS
%----------------------------------------------------------------------------------------
% \section{Publications}
% \begin{refsection}[citations.bib]
% \nocite{*}
% \printbibliography[heading=none]
% \end{refsection}

%----------------------------------------------------------------------------------------
%	SKILLS
%----------------------------------------------------------------------------------------
\section{Skills}
\begin{tabularx}{\linewidth}{@{}l X@{}}
\textbf{Languages:} & \normalsize{JavaScript, Dart, Python, C, R} \\
\textbf{Frameworks/Libraries:} & \normalsize{Express.js, React.js, Next.js, Node.js, Flutter, Flask, Tailwind CSS} \\
\textbf{Databases:} & \normalsize{MongoDB, Postgresql, MySQL} \\
\textbf{Operating Systems:} & \normalsize{Linux} \\
\end{tabularx}

\vfill
\center{\footnotesize Last updated: \today}

\end{document}
